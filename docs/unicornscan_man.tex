\documentclass[english]{article}

\usepackage[latin1]{inputenc}
\usepackage{latex2man}
\usepackage{babel}
\usepackage{verbatim}

% kekekekekekee
\newcommand{\Sysconfdir}{/usr/local/etc}


\setDate{03/30/05}
\setVersion{0.4.6b}

\begin{document}

% \begin{Name}{chapter}{name}{author}{info}{title}
\begin{Name}{1}{unicornscan}{Jack}{Network Tools}{unicornscan command documentation}
%%%%%%%%%%

\Prog{unicornscan} Version \Version\ is a asynchronous network stimulus delivery/response recoring
tool.

\end{Name}

\section{Synopsis}
%%%%%%%%%%

\Prog{unicornscan}
\oOptArg{-b, --broken-crc    }{layer}
\oOptArg{-B, --source-port   }{port}
\oOptArg{-d, --delay-type    }{type}
   \oOpt{-D, --no-defpayload }
\oOptArg{-e, --enable-module }{modules}
   \oOpt{-E, --proc-errors   }
   \oOpt{-F, --try-frags     }
\oOptArg{-G, --payload-group }{group}
   \oOpt{-h, --help          }
   \oOpt{-H, --do-dns        }
\oOptArg{-i, --interface     }{interface}
   \oOpt{-I, --immediate     }
\oOptArg{-j, --ignore-seq    }{ignore}
\oOptArg{-l, --logfile       }{file}
\oOptArg{-L, --packet-timeout}{delay}
\oOptArg{-m, --mode          }{mode}
\oOptArg{-M, --module-dir    }{directory}
\oOptArg{-p, --ports         }{string}
\oOptArg{-P, --pcap-filter   }{filter}
\oOptArg{-q, --covertness    }{covertness}
   \oOpt{-Q, --quiet         }
\oOptArg{-r, --pps           }{rate}
\oOptArg{-R, --repeats       }{repeats}
\oOptArg{-s, --source-addr   }{address}
   \oOpt{-S, --no-shuffle    }
\oOptArg{-t, --ip-ttl        }{TTL}
\oOptArg{-T, --ip-tos        }{TOS}
\oOptArg{-w, --safefile      }{file}
\oOptArg{-W, --fingerprint   }{fingerprint}
   \oOpt{-v, --verbose       }
   \oOpt{-V, --version       }
   \oOpt{-z, --sniff         }
\oOptArg{-Z, --drone-type    }{type}
    \Arg{target list}

\section{Description}
%%%%%%%%%%

\Prog{unicornscan}:
	...

\section{Options}
%%%%%%%%%%
\begin{Description}
\item[\oOptArg{-b, --broken-crc      }{Layer}]
Break CRC sums on the following layers. N and T are valid, and both may be used without separator,
so NT would indicate both Network and Transport layers are to have invalid checksums.
\item[\oOptArg{-B, --source-port     }{Port}]
Source port for sent packets, numeric value -1 means to use a random source port (the default situation),
and other valid settings are 0 to 65535. normally this option will not be used, but sometimes it is useful to say
scan from port 53 into a network.
\item[\oOptArg{-d, --delay-type      }{Type}]
Specify the timer used for pps calculations, the default is variable and will try and use something appropriate
for the rate you have selected. Note however, if available, the tsc timer and the gtod timer are very CPU intensive.
if you require unicornscan to not monopolize your system while running, consider using the sleep timer, normally 3.
it has been observed that the tsc timer and gtod timer are required for high packet rates, however this is highly
system dependent, and should be tested on each hardware/platform combination. The tsc timer may not be available
on every cpu. The sleep timer module is not recommended for scans where utmost accuracy is required.
\item[\oOpt{-D, --no-defpayload   }]
Do not use default payloads when one cannot be found.
\item[\oOptArg{-e, --enable-module   }{List}]
A comma separated list of modules to activate (note: payload modules do not require explicit activation, as they are
enabled by default). an example would be `pgsqldb,foomod'.
\item[   \oOpt{-E, --proc-errors     }]
Enable processing of errors such as icmp error messages and reset+ack messages (for example). If this option is set
then you will see responses that may or may not indicate the presence of a firewall, or other otherwise missed
information.
\item[   \oOpt{-F, --try-frags       }]
It is likely that this option doesn't work, don't bother using it until it is fixed.
\item[\oOptArg{-G, --payload-group   }{Group}]
activate payloads only from this numeric payload group. The default payload group is 1.
\item[   \oOpt{-h, --help            }]
if you don't know what this means, perhaps you should consider not using this program.
\item[   \oOpt{-H, --do-dns          }]
Resolve dns hostnames before and after the scan (but not during, as that would likely cause superfluous spurious
responses during the scan, especially if udp scanning). the hosts that will be resolved are (in order of resolution)
the low and high addresses of the range, and finally each host address that replied with something that would be
visible depending on other scan options. This option is not recommended for use during scans where utmost accuracy
is required.
\item[\oOptArg{-i, --interface       }{Interface}]
string representation of the interface to use, overriding automatic detection.
\item[   \oOpt{-I, --immediate       }]
Display results immediately as they are found in a sort of meta report format (read: terse). This option is not
recommended for use during scans where the utmost accuracy is required.
\item[\oOptArg{-j, --ignore-seq      }{Type}]
A string representing the intended sequence ignorance level. This affects the tcp header validity checking, normally
used to filter noise from the scan. If for example you wish to see reset packets with an ack+seq that is not set
or perhaps intended for something else appropriate use of this option would be R. A is normally used for more exotic
tcp scanning. normally the R option is associated with reset scanning.
\item[\oOptArg{-l, --logfile         }{File}]
Path to a file where flat text will be dumped that normally would go to the users terminal. A limitation of this option
currently is that it only logs the output of the `Main' thread and not the sender and receiver.
\item[\oOptArg{-L, --packet-timeout  }{Seconds}]
Numeric value representing the number of seconds to wait before declaring the scan over. for connect scans sometimes
this option can be adjusted to get more accurate results, or if scanning a high-latency target network; for example.
\item[\oOptArg{-m, --mode            }{Mode}]
String representation of the desired scanning mode. Correct usage includes U, T, A and sf for Udp scanning, Tcp scanning, Arp scanning, and Tcp Connect scanning respectively.
\item[\oOptArg{-M, --module-dir      }{Directory}]
Path to a directory containing shared object `modules' for unicornscan to search.
\item[\oOptArg{-p, --ports           }{Ports}]
A global list of ports to scan, can be overridden in the target specification on a per target basis.
\item[\oOptArg{-P, --pcap-filter     }{Filter}]
A pcap filter string to add to the listeners default pcap filter (that will be associated with the scan mode being used).
\item[\oOptArg{-c, --covertness      }{Level}]
Numeric option that currently does nothing, except look cool.
\item[   \oOpt{-Q, --quiet           }]
This option is intended to make unicornscan play the `quiet game'. If you are unfamiliar with its rules, consult with
someone else who finds you irritating.
\item[\oOptArg{-r, --pps             }{Rate}]
This is arguably the most important option, it is a numeric option containing the desired packets per second for the
sender to use. choosing a rate too high will cause your scan results to be incomplete. choosing a rate too low will
likely make you feel as though you are using nmap.
\item[\oOptArg{-R, --repeats         }{Times}]
The number of times to completely repeat the senders workload, this option is intended to improve accuracy during
critical scans, or with scans going over a highly unreliable network.
\item[\oOptArg{-s, --source-addr     }{Address}]
The address to use to override the listeners default interfaces address. using this option often necessitates using
the helper program \Cmd{fantaip}{1} to make sure the replies are routed back to the interface the listener has open.
\item[   \oOpt{-S, --no-shuffle      }]
..
\item[\oOptArg{-t, --ip-ttl          }{Number}]
..
\item[\oOptArg{-T, --ip-tos          }{Number}]
..
\item[\oOptArg{-w, --savefile        }{File}]
..
\item[\oOptArg{-W, --fingerprint     }{Type}]
..
\item[   \oOpt{-v, --verbose         }]
..
\item[   \oOpt{-V, --version         }]
..
\item[   \oOpt{-z, --sniff           }]
..
\item[\oOptArg{-Z, --drone-type      }{Type}]
..
\end{Description}

\section{Examples}
%%%%%%%%%%
\begin{Description}
\textbf{unicornscan -msf -s 5.4.3.2 -r 340 -Iv -epgsqldb www.domain.tld/21:80,8080,443,81}
runs unicornscan in connect mode with an apparent (to the target) source address of 5.4.3.2 at a rate
of 340 packets per second. results will be displayed as they are found \emph{-I} and the output
will be verbose \emph{-v}. The module `pgsqldb' will be activated \emph{-epgsqldb} and
the target of this scan will be the /21 network that host www.domain.tld belongs to making attempts
to connect to port 80, 8080, 443 and 81.
\end{Description}

\section{Files}
%%%%%%%%%%
\begin{Description}
\item[\File{unicorn.conf}] The file containing the default configuration options for usage.
\item[\File{modules.conf}] The default file for module parameters.
\item[\File{oui.txt}] Contains the MAC prefix to vendor mapping used in Ethernet scanning.
\item[\File{payloads.conf}] The default file for tcp and udp payloads.
\item[\File{ports.txt}] The protocol/port number to name mapping.
\end{Description}

\section{See Also}
%%%%%%%%%%
\Cmd{fantaip}{1} \Cmd{unicfgtst}{1} \Cmd{unicycle}{1} \Cmd{unibrow}{1} \Cmd{unicorn.conf}{5}

\section{Reporting Bugs}
%%%%%%%%%%
\begin{description}
Report Bugs to osace-users@lists.sourceforge.net
\end{description}

\section{Copyright}
%%%%%%%%%%
\begin{description}
\copyright\ 2004 Jack Louis \Email{jack@rapturesecurity.org}
This is free software; see the source for  copying  conditions.  There is NO warranty; not even for
MERCHANTABILITY or FITNESS FOR A PARTICULAR PURPOSE.
\end{description}

\LatexManEnd

\end{document}
